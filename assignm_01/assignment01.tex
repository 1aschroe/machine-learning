\documentclass[fontsize=12pt,a4paper]{scrartcl}
 
% Das Prozent-Zeichen leitet einen Kommentar ein,
% es hilft ebenso, im Text Leerzeichen zu unterbinden.
 
% fontsize=12pt  Schriftgroesse in 10, 11 oder 12 Punkt
% a4paper        Papierformat ist hier A4
% landscape      Querformat wird natürlich unterstützt ;-)
% parskip        Absatzabstand anstatt Einzüge
% draft          Der Entwurfsmodus deckt Schwächen auf
% {scrartcl}     Die Dokumentenklasse book, report, article
%                oder fürs deutsche scrbook, scrreprt, scrartcl
 
%\usepackage[ngerman]{babel} % Deutsche Sprachanpassungen
\usepackage[T1]{fontenc}    % Silbentrennung bei Sonderzeichen
\usepackage[latin1]{inputenc} % Direkte Angabe von Umlauten im Dokument.
                            % Wenn Sie an einem Mac sitzen,verwenden
                            % Sie ggf. „macce“ anstatt „utf8“.
 
\usepackage{textcomp}       % Zusätzliche Symbolzeichen
\usepackage{siunitx}
\usepackage{amsmath}
\usepackage{listings}
\lstset{tabsize=4, showspaces=false}


\title{Machine Learning SS2013}
\subtitle{Ulrike von Luxburg \\ Assignment 01}
\author{Arne Schr�der, Falk Oswald, Angel Bakardzhiev}
 
\date{\today}               % \today setzt das heutige Datum
 
\begin{document}
\maketitle                  % Titelei erzeugen
% \tableofcontents            % Inhaltsverzeichnis anlegen
 
 
\section*{Matlab Implementation}
First, we introduce and briefly describe our M files, included in the attached zip file.

\begin{itemize}
	\item \textbf{knnClassifySingle.m} - function, that uses k-nearest neighbours method to predict label of single datum
	\item \textbf{knnClassify.m} - function, that uses k-nearest neighbours method to predict labels
	\item \textbf{evaluateK.m} - evaluates knnClassify for different k-values and returns the minimal k
	\item \textbf{loss01.m} - Gets as input a prediction calculated by the knnClasifiy and correct labels y. The function returns the average error (empirical risk with respect to the 0-1 loss) for this prediction.
	\item \textbf{drawNumber.m} - visualize a number using \textit{imagesc}
	\item \textbf{doExercise1.m} - loads all training and test data for exercise 1, calls knnClassify and plots the result
	\item \textbf{doExercise2.m} - loads all training and test data for exercise 2, calls knnClassify and plots the result
	\item \textbf{Assignment01.m} - the main script, calls doExercise1 and doExercise2 with different parameters
\end{itemize}

\section*{Questions}

 \paragraph{1.7. Plot the training and the test errors. Do results change between different runs? Why?} \hfill
 
 Yes, the results change between different runs. The reason is, that we use random training and test data. For each run the data is different, so we get different results.
 
 \paragraph{1.9. More training examples. How does the performance of kNN classifier change?} \hfill
 
 The performance of the classifier is the same like before for the test data, increases however approximately by factor 10 for the training data.
 
 \paragraph{1.10. Unbalanced classes. How does the performance of kNN classifier change?} \hfill
 
 The error of the classifier increases approximately by factor $1/2$ for the training data and by factor 40 for the test data.
 
 \paragraph{2.5. Run your algorithm to classify digit 3 from 8 and compare its performance with results from digit 2 versus 3.} \hfill
 
 The results from both runs are similar in general. Over all the classification between 3 and 8 has errors higher by factor 3.
 
\end{document}