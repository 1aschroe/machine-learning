\documentclass[fontsize=12pt,a4paper]{scrartcl}
 
% Das Prozent\item Zeichen leitet einen Kommentar ein,
% es hilft ebenso, im Text Leerzeichen zu unterbinden.
 
% fontsize=12pt  Schriftgroesse in 10, 11 oder 12 Punkt
% a4paper        Papierformat ist hier A4
% landscape      Querformat wird natürlich unterstützt ;\item )
% parskip        Absatzabstand anstatt Einzüge
% draft          Der Entwurfsmodus deckt Schwächen auf
% {scrartcl}     Die Dokumentenklasse book, report, article
%                oder fürs deutsche scrbook, scrreprt, scrartcl
 
%\usepackage[ngerman]{babel} % Deutsche Sprachanpassungen
\usepackage[T1]{fontenc}    % Silbentrennung bei Sonderzeichen
\usepackage[latin1]{inputenc} % Direkte Angabe von Umlauten im Dokument.
                            % Wenn Sie an einem Mac sitzen,verwenden
                            % Sie ggf. „macce“ anstatt „utf8“.
 
\usepackage{textcomp}       % Zusätzliche Symbolzeichen
\usepackage{siunitx}
\usepackage{amsmath}
\usepackage{graphicx}
\usepackage{listings}
\usepackage{float}
\lstset{tabsize=4, showspaces=false}


\title{Machine Learning SS2013}
\subtitle{Ulrike von Luxburg \\ Assignment 11}
\author{Arne Schr�der \and Falk Oswald \and Angel Bakardzhiev}
 
\date{\today}               % \today setzt das heutige Datum
 
\begin{document}
\maketitle                  % Titelei erzeugen
% \tableofcontents            % Inhaltsverzeichnis anlegen

\section*{Exercise 1: Beispiel f�r Anwendung von Reinforcement Learning}

\subsection*{Automatischer Pfannkuchenwender}

\begin{description}
\item [States] Positionen, Geschwindigkeit von Pfanne und Pfannkuchen
\item [Actions] �nderung der Geschwindikeit der Pfanne
\item [Rewards-Funktion] Entfernung der Lande-Position des Pfannkuchens vom Zentru der Pfanne. Drehung des Pfannkuchens (Hat er sich gedreiht, wie oft?): Position geht �ber Fau�-Funktion mit kleinem Faktor ein. Einmaliges Wenden bringt 1000, zweimaliges Wenden 1100 usw.
\item [State-Space] 3d, Drehung: 3d, Geschwindigkeit:, das ganze f�r Pfanne und Kuchen: 24d.
\end{description}

\subsection*{Automatischer Staubsaugerroboter}

\begin{description}
\item [States] Position, Ausrichtung (Orientierung), abgefahrene Fl�che
\item [Actions] Zu n�chster Position fahren
\item [Rewards-Funktion] Menge des �brig gebliebenen Staubs stark negativ ber�cksichtigen und die Zeit, die ben�tigt wird, leicht negativ.
\item [State-Space] Position: 2d, Orientierung: 1d, Abgefahrene Fl�che (n*m Rasterpunkte (0= nicht besucht, 1 = besucht)
\end{description}

\subsection*{Lernende K�nstliche Intelligenz f�r Schach}
\begin{description}
\item [States] Positionen der Figuren
\item [Actions] Bewegen der Figuren
\item [Rewards-Funktion] Gewichtete Anzahl der Figuren, Bewegungsm�glichkeiten, Schach(matt)
\item [State-Space] 64d (F�r jedes Feld, welche Figur drauf steht)
\end{description}

\section*{RL-Ansatz}
Nein.

RL ist nicht angebracht, wenn 100 \% zuverl�ssige Anforderungen gestellt werden (bsp. Flugzeug fliegen), da es keine Garantie gibt, dass RL zu irgend einem zeitpunkt nicht scheitern wird.

RL kann ebenfalls nicht angewendet werden, wenn ein State nicht eindeutigt bestimmt werden kann oder wenn keine gute Reward Funktion definiert werden kann.

\section*{Exercise 2: Grid-world V-function}

\subsection*{Optimale Policy}
See slide 66 c)

\section*{Exercise 3: Bellman Equation for the $Q$-function}

$Q^\pi(s, a)$ gives the value of an action $a$ when being in state $s$ using the policy $\pi$.

To calculate this, we consider all possible states, that follow $s'$ with their probability $P^a_{ss'}$.

\[
  Q^\pi(s, a) = \sum_{s'} P^a_{ss'} (\dots)
\]

For each of these follow up states, we consider the expected rewards $R^a_{ss'}$ and the value which could be obtained in following actions $a'$: $Q^\pi(s', a')$. These are weighted by the policy $\pi$ given:

\[
  Q^\pi(s, a) = \sum_{s'} P^a_{ss'} \left(R^a_{ss'} + \sum_{a'} \pi(s', a') Q^\pi(s', a') \right)
\]

\section*{Exercise 4: Maze}

Wird f�r jeden beliebigen Durchlauf durch das Labyrinth der gleiche Reward vergeben, so ist in diesem Sinne auch jeder Durchlauf gleich gut (solange er nicht in einer Endlosschleife h�ngen bleibt). Wir w�rden jedoch einen zielstrebigen Durchlauf, welcher weniger Schritte ben�tigt, als besser ansehen.

Daher muss hier die Reward-Funktion dahingehend ge�ndert werden, dass l�ngere Durchl�ufe weniger Reward erbringen als kurze. Beispielsweise dadurch, dass die +1 durch die Anzahl der Schritte geteilt wird, oder aber eine fixe Zahl (beispielsweise 100) um die Anzahl der Schritte verringert wird.
\end{document}