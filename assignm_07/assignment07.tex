\documentclass[fontsize=12pt,a4paper]{scrartcl}
 
% Das Prozent\item Zeichen leitet einen Kommentar ein,
% es hilft ebenso, im Text Leerzeichen zu unterbinden.
 
% fontsize=12pt  Schriftgroesse in 10, 11 oder 12 Punkt
% a4paper        Papierformat ist hier A4
% landscape      Querformat wird natürlich unterstützt ;\item )
% parskip        Absatzabstand anstatt Einzüge
% draft          Der Entwurfsmodus deckt Schwächen auf
% {scrartcl}     Die Dokumentenklasse book, report, article
%                oder fürs deutsche scrbook, scrreprt, scrartcl
 
%\usepackage[ngerman]{babel} % Deutsche Sprachanpassungen
\usepackage[T1]{fontenc}    % Silbentrennung bei Sonderzeichen
\usepackage[latin1]{inputenc} % Direkte Angabe von Umlauten im Dokument.
                            % Wenn Sie an einem Mac sitzen,verwenden
                            % Sie ggf. „macce“ anstatt „utf8“.
 
\usepackage{textcomp}       % Zusätzliche Symbolzeichen
\usepackage{siunitx}
\usepackage{amsmath}
\usepackage{listings}
\lstset{tabsize=4, showspaces=false}


\title{Machine Learning SS2013}
\subtitle{Ulrike von Luxburg \\ Assignment 07}
\author{Arne Schr�der \and Falk Oswald \and Angel Bakardzhiev}
 
\date{\today}               % \today setzt das heutige Datum
 
\begin{document}
\maketitle                  % Titelei erzeugen
% \tableofcontents            % Inhaltsverzeichnis anlegen

 \section*{Exercise 1}
F�r die Entscheidung zur Wahl eines Klassifikators spielen verschiedene Faktoren eine Rolle, die im Folgenden aufgelistet werden.

Indirekte Faktoren:
\begin{itemize}
\item Preis
\item Performance
\item Skalierung
\item Komplexit�t (der Benutzung)
\item Dokumentation
\end{itemize}

Direkte Faktoren:
\begin{itemize}
\item Gesamtqualit�t (geringe Fehlerquote)
\item Fehleranf�lligkeit (bsp. outlier detection)
\item Einstellbarkeit (Geringe Komplexit�t bei der Findung der besten Einstellungen)
\item In Abh�ngigkeit von den Daten, einen passenden Klassifikator �ber den bias ausw�hlen (ein linearer Klassifikator macht bspws. bei konzentrischen Daten keinen Sinn)
\end{itemize}

Setup zum Pr�fen von Klassifikatoren
\begin{itemize} 
\item Zun�chst repr�sentative Daten erfassen
\item Immer gleiche Trainings\item  und Testdaten verwenden
\item Die M�chtigkeit der Tranings- und Testdaten und deren Verh�ltnis variieren
\item M�glichst viele (statistisch unabh�ngige) Trainings\item  und Testdaten
\item Sukzessive Erweiterung des Testsatzes um eventuelle Skalierungsfaktoren vorhersagen zu k�nnen
\end{itemize}

Evaluierung der Klassifikatoren nach
\begin{itemize} 
\item False positive/negative rate
\item Leistung (Zeit und Platzverbrauch)
\item Output der Interpretation des Algorithmus
\end{itemize}

\section*{Exercise 2}

\textbf{1. Selecting candidate classifiers: Consider at least three different classifiers}\\

Our classifiers:
\begin{itemize} 
\item kNN
\item SVM
\item LDA
\end{itemize}

\textbf{2. Choosing datasets: You should use at least three different datasets.}\\

The datasets that we used:
\begin{itemize} 
\item USPS
\item Breast cancer
\item Shuttle dataset
\end{itemize}

\textbf{3. Run the algorithms on the data sets and compare the results.}\\


\textbf{4. Try to understand the results and the data.}\\

\textbf{5. Visualization}\\

See the Matlab plots.

\end{document}
