\documentclass[fontsize=12pt,a4paper]{scrartcl}
 
% Das Prozent\item Zeichen leitet einen Kommentar ein,
% es hilft ebenso, im Text Leerzeichen zu unterbinden.
 
% fontsize=12pt  Schriftgroesse in 10, 11 oder 12 Punkt
% a4paper        Papierformat ist hier A4
% landscape      Querformat wird natürlich unterstützt ;\item )
% parskip        Absatzabstand anstatt Einzüge
% draft          Der Entwurfsmodus deckt Schwächen auf
% {scrartcl}     Die Dokumentenklasse book, report, article
%                oder fürs deutsche scrbook, scrreprt, scrartcl
 
%\usepackage[ngerman]{babel} % Deutsche Sprachanpassungen
\usepackage[T1]{fontenc}    % Silbentrennung bei Sonderzeichen
\usepackage[latin1]{inputenc} % Direkte Angabe von Umlauten im Dokument.
                            % Wenn Sie an einem Mac sitzen,verwenden
                            % Sie ggf. „macce“ anstatt „utf8“.
 
\usepackage{textcomp}       % Zusätzliche Symbolzeichen
\usepackage{siunitx}
\usepackage{amsmath}
\usepackage{graphicx}
\usepackage{listings}
\lstset{tabsize=4, showspaces=false}


\title{Machine Learning SS2013}
\subtitle{Ulrike von Luxburg \\ Assignment 09}
\author{Arne Schr�der \and Falk Oswald \and Angel Bakardzhiev}
 
\date{\today}               % \today setzt das heutige Datum
 
\begin{document}
\maketitle                  % Titelei erzeugen
% \tableofcontents            % Inhaltsverzeichnis anlegen

\section*{Exercise 3: Spectral clustering demo}

\subsection*{Question}

For which range of $k$ do the clusters in the embedding look ``well separated''?

\subsection*{Answer}

For the balanced two moons data set, best values were obtained when using a $k$ between 4 and 20.

\subsection*{Question}

Describe the interplay between the kernel width $\sigma$ and the number of neighbours $k$. What would happen if we choose a large $k$, but small $\sigma$?

\subsection*{Answer}

If $k$ becomes too big, $\sigma$ becomes more important in separating clusters: If both values are high, every point is seen as being similar to every other point, thus reducing the algorithm's ability to distinguish between clusters. If we chose $\sigma$ small, $k$ becomes less relevant, as many points are connected by a value close to zero.

If we chose a big $k$ and a small $\sigma$, the algorithm would become very sensitive to outliers, because then the distance between points is the main criterion for distinguishing clusters, making spectral clustering similar valuable as single linkage.
\end{document}
