\documentclass[fontsize=12pt,a4paper]{scrartcl}
 
% Das Prozent-Zeichen leitet einen Kommentar ein,
% es hilft ebenso, im Text Leerzeichen zu unterbinden.
 
% fontsize=12pt  Schriftgroesse in 10, 11 oder 12 Punkt
% a4paper        Papierformat ist hier A4
% landscape      Querformat wird natürlich unterstützt ;-)
% parskip        Absatzabstand anstatt Einzüge
% draft          Der Entwurfsmodus deckt Schwächen auf
% {scrartcl}     Die Dokumentenklasse book, report, article
%                oder fürs deutsche scrbook, scrreprt, scrartcl
 
%\usepackage[ngerman]{babel} % Deutsche Sprachanpassungen
\usepackage[T1]{fontenc}    % Silbentrennung bei Sonderzeichen
\usepackage[latin1]{inputenc} % Direkte Angabe von Umlauten im Dokument.
                            % Wenn Sie an einem Mac sitzen,verwenden
                            % Sie ggf. „macce“ anstatt „utf8“.
 
\usepackage{textcomp}       % Zusätzliche Symbolzeichen
\usepackage{siunitx}
\usepackage{amsmath}
\usepackage{listings}
\usepackage{graphicx}
\lstset{tabsize=4, showspaces=false}


\title{Machine Learning SS2013}
\subtitle{Ulrike von Luxburg \\ Assignment 01}
\author{Arne Schr�der, Falk Oswald, Angel Bakardzhiev}
 
\date{\today}               % \today setzt das heutige Datum
 
\begin{document}
\maketitle                  % Titelei erzeugen
% \tableofcontents            % Inhaltsverzeichnis anlegen
 
 
\section*{Matlab Implementation}
First, we introduce and briefly describe our M files, included in the attached zip file.

\begin{itemize}
	\item \textbf{knnClassifySingle.m} - function, that uses k-nearest neighbours method to predict label of single datum
	\item \textbf{knnClassify.m} - function, that uses k-nearest neighbours method to predict labels
	\item \textbf{evaluateK.m} - evaluates knnClassify for different k-values and returns the minimal k
	\item \textbf{loss01.m} - Gets as input a prediction calculated by the knnClasifiy and correct labels y. The function returns the average error (empirical risk with respect to the 0-1 loss) for this prediction.
	\item \textbf{drawNumber.m} - visualize a number using \textit{imagesc}
	\item \textbf{doExercise1.m} - loads all training and test data for exercise 1, calls knnClassify and plots the result
	\item \textbf{doExercise2.m} - loads all training and test data for exercise 2, calls knnClassify and plots the result
	\item \textbf{Assignment01.m} - the main script, calls doExercise1 and doExercise2 with different parameters
\end{itemize}

\section*{Questions}

 \paragraph{1.7. Plot the training and the test errors. Do results change between different runs? Why?} \hfill
 
 Yes, the results change between different runs. The reason is, that we use random training and test data. For each run the data is different, so we get different results.
 
 \paragraph{1.9. More training examples. How does the performance of kNN classifier change?} \hfill
 
 The performance of the classifier is the same like before for the test data, increases however approximately by factor 10 for the training data.
 
 \paragraph{1.10. Unbalanced classes. How does the performance of kNN classifier change?} \hfill
 
 The error of the classifier increases approximately by factor $1/2$ for the training data and by factor 40 for the test data.
 
 \paragraph{2.5. Run your algorithm to classify digit 3 from 8 and compare its performance with results from digit 2 versus 3.} \hfill
 
 The results from both runs are similar in general. Over all the classification between 3 and 8 has errors higher by factor 3.
 
\section*{Exercise 5}

\subsection*{Question 1}

What are false positive, false negative and average error of you(r) calssifier?

\subsection*{Answer}

To answer this question we simply need to fill out the following table:

\begin{tabular}{c||c|c|c}
& spam & not spam & overall \\ 
\hline\hline
 classified as spam &  &  &  \\ 
\hline
 classified as not spam &  &  &  \\ 
\hline
 overall & 60 \% &  & 
\end{tabular} 

We know that 85 \% of spam is classified as such, which gives us that $60 \% \cdot 85 \% = 51 \%$ of mails are spam and are classified as spam.

As 60 \% of all mail is spam, 40 \% of mail is not. This means that if 5 \% of all non-spam mails are classified as spam this is 2 \% of all mails.

\begin{tabular}{c||c|c|c}
& spam & not spam & overall \\ 
\hline\hline
 classified as spam & 51 \% & 2 \% &  \\ 
\hline
 classified as not spam &  &  &  \\ 
\hline
 overall & 60 \% & 40 \% & 
\end{tabular}

Subtraction now tells us that 9 \% of mails are spam but not classified spam and 38 \% are spam and correctly classified.

\begin{tabular}{c||c|c||c}
& spam & not spam & overall \\ 
\hline\hline
 classified as spam & 51 \% & 2 \% & 53 \% \\ 
\hline
 classified as not spam & 9 \% & 38 \% & 47 \% \\ 
\hline \hline
 overall & 60 \% & 40 \% & 100 \%
\end{tabular}

Therefore, the false positive rate is 2 \% and the false negative rate is 9 \%.

\section*{Task 2}

Find a classification algorithm with false positive rate 0. Find a classification algorithm with false negative rate 0.

\section*{Solution}

An algorithm with false positive rate of 0 can be achieved by not classifying anything as spam (resulting in a 40 \% false negative rate).

Conversely, an algorithm with false negative rate of 0 can be achieved by classifying everything as spam (resulting in a 60 \% false positive rate).

\section*{Question 3}

Which entries of $X'$ lead to a false positive error and which ones to a false negative error in the Bayes classification?

\section*{Answer}

The false positives are those classified as 2 but being 1, in this case this applies to 2.5. The false negatives are those classified as 1 but being 2, in this case this applies to 2.

\section*{Task 4}

Sketch the approximate Bayesian decision boundary by hand with respect to the following loss functions
\begin{itemize}
\item[-] 0-1 loss
\item[-] Unsymmetric los(s): $\ell(\text{spam},\text{non-spam}) = 1$, $\ell(\text{non-spam},\text{spam}) = 100$.
\end{itemize}

\section*{Solution}

The first is the same as if there was no loss function. In the following graph, one can see a with solid lines the decision curves for none or a 0-1 loss function and with dashed lines are the decision curves for the asymmetric loss. Magenta being in favour for 1 (non-spam) and black being in favour for 2 (spam):
\begin{center}
  \includegraphics[width=\textwidth]{decision_graph.png}
\end{center}

\end{document}