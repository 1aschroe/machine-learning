\documentclass[fontsize=12pt,a4paper]{scrartcl}
 
% Das Prozent-Zeichen leitet einen Kommentar ein,
% es hilft ebenso, im Text Leerzeichen zu unterbinden.
 
% fontsize=12pt  Schriftgroesse in 10, 11 oder 12 Punkt
% a4paper        Papierformat ist hier A4
% landscape      Querformat wird natürlich unterstützt ;-)
% parskip        Absatzabstand anstatt Einzüge
% draft          Der Entwurfsmodus deckt Schwächen auf
% {scrartcl}     Die Dokumentenklasse book, report, article
%                oder fürs deutsche scrbook, scrreprt, scrartcl
 
%\usepackage[ngerman]{babel} % Deutsche Sprachanpassungen
\usepackage[T1]{fontenc}    % Silbentrennung bei Sonderzeichen
\usepackage[latin1]{inputenc} % Direkte Angabe von Umlauten im Dokument.
                            % Wenn Sie an einem Mac sitzen,verwenden
                            % Sie ggf. „macce“ anstatt „utf8“.
 
\usepackage{textcomp}       % Zusätzliche Symbolzeichen
\usepackage{siunitx}
\usepackage{amsmath}
\usepackage{listings}
\lstset{tabsize=4, showspaces=false}


\title{Machine Learning SS2013}
\subtitle{Ulrike von Luxburg \\ Assignment 03}
\author{Arne Schr�der \and Falk Oswald \and Angel Bakardzhiev}
 
\date{\today}               % \today setzt das heutige Datum
 
\begin{document}
\maketitle                  % Titelei erzeugen
% \tableofcontents            % Inhaltsverzeichnis anlegen
 
\section*{Exercise 1 and 2}
See Matlab files in the attached zip file.

\section*{Exercise 3}
Compare computation time of the least square regression and ridge regression.

\subsection*{Answer 3}
While the linear least square method (LLS) uses a average computation time of about 0.2 millisec, the ridge regression algorithm (RidgeLLS) takes nearly twice the time (an average of 0.4 msec).
 

\section*{Exercise 4}

\subsection*{Question}
Prove the least squares loss function $\| Y-X w \|^2$ is a convex function of $w$.

\subsection*{Answer}

\begin{align*}
  &\| Y-X (tw_1 + (1-t)w_2) \|^2  \\
  =& \langle Y-X (tw_1 + (1-t)w_2), Y-X (tw_1 + (1-t)w_2) \rangle \\
  =& \langle t(Y-X w_1) + (1-t)(Y-X w_2), t(Y-X w_1) + (1-t)(Y-X w_2) \rangle \\
  =& \langle t(Y-X w_1), t(Y-X w_1) + (1-t)(Y-X w_2) \rangle + \langle (1-t)(Y-X w_2), t(Y-X w_1) \; + \\
  & (1-t)(Y-X w_2) \rangle \\
  =& \langle t(Y-X w_1), t(Y-X w_1) \rangle + \langle t(Y-X w_1), (1-t)(Y-X w_2) \rangle \; + \\
  & \langle (1-t)(Y-X w_2), t(Y-X w_1) \rangle + \langle (1-t)(Y-X w_2), (1-t)(Y-X w_2) \rangle \\
  =& t^2\langle Y-X w_1, Y-X w_1 \rangle + 2 t (1-t)\langle Y-X w_1, Y-X w_2 \rangle + (1-t)^2\langle Y-X w_2, Y-X w_2 \rangle 
\end{align*}

\end{document}