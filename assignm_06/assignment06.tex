\documentclass[fontsize=12pt,a4paper]{scrartcl}
 
% Das Prozent-Zeichen leitet einen Kommentar ein,
% es hilft ebenso, im Text Leerzeichen zu unterbinden.
 
% fontsize=12pt  Schriftgroesse in 10, 11 oder 12 Punkt
% a4paper        Papierformat ist hier A4
% landscape      Querformat wird natürlich unterstützt ;-)
% parskip        Absatzabstand anstatt Einzüge
% draft          Der Entwurfsmodus deckt Schwächen auf
% {scrartcl}     Die Dokumentenklasse book, report, article
%                oder fürs deutsche scrbook, scrreprt, scrartcl
 
%\usepackage[ngerman]{babel} % Deutsche Sprachanpassungen
\usepackage[T1]{fontenc}    % Silbentrennung bei Sonderzeichen
\usepackage[latin1]{inputenc} % Direkte Angabe von Umlauten im Dokument.
                            % Wenn Sie an einem Mac sitzen,verwenden
                            % Sie ggf. „macce“ anstatt „utf8“.
 
\usepackage{textcomp}       % Zusätzliche Symbolzeichen
\usepackage{siunitx}
\usepackage{amsmath}
\usepackage{listings}
\lstset{tabsize=4, showspaces=false}


\title{Machine Learning SS2013}
\subtitle{Ulrike von Luxburg \\ Assignment 06}
\author{Arne Schr�der \and Falk Oswald \and Angel Bakardzhiev}
 
\date{\today}               % \today setzt das heutige Datum
 
\begin{document}
\maketitle                  % Titelei erzeugen
% \tableofcontents            % Inhaltsverzeichnis anlegen

 \section*{Exercise 1}
	Best parameter for example with noise is $c=1$.

	%TODO: include figures and comments
	
 \section*{Exercise 2}
    Figure 2 a is most likely to be a Gaussian kernel (rbf) with parameters $\beta = 10$.
	
	One could argue, that this scenario is already over-fitting as a linear kernel would only lead to two false positives but a much simpler solution. On the other hand, the Gaussian kernel does classify every point correctly and provides a not too complex function. A change in parameters is not essentially necessary.
	
	Figure 2 b is most likely to be a polynomial kernel with parameter $c = 10$.
	
	The chosen scenario does not separate the classes correctly. Points, which are most definitely no outliers (as they lie in groups) are classified wrong. The situation can be made better by raising $c$ to 100 and the degree $d$ to 3.
	
	We guess that almost all points are support vectors. Maybe some of the points in between other points of the same class are not.
 \end{document}