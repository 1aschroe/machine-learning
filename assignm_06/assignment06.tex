\documentclass[fontsize=12pt,a4paper]{scrartcl}
 
\usepackage[T1]{fontenc}    
\usepackage[latin1]{inputenc} 
 
\usepackage{textcomp} 
\usepackage{amsmath}
\usepackage{amssymb}
\usepackage{listings}
\lstset{tabsize=4, showspaces=false}


\title{Machine Learning SS2013}
\subtitle{Ulrike von Luxburg \\ Assignment 06}
\author{Arne Schr�der \and Falk Oswald \and Angel Bakardzhiev}
 
\date{\today} 
 
\begin{document}
\maketitle                  

 \section*{Exercise 1}
	Best parameter for example with noise is $c=1$.

	%TODO: include figures and comments
	
 \section*{Exercise 2}
    Figure 2 a is most likely to be a Gaussian kernel (rbf) with parameters $\beta = 10$.
	
	One could argue, that this scenario is already overfitting as a linear kernel would only lead to two false positives but a much simpler solution. On the other hand, the Gaussian kernel does classify every point correctly and provides a not too complex function. A change in parameters is not essentially neccesary.
	
	Figure 2 b is most likely to be a polynomial kernel with parameter $c = 10$.
	
	The chosen scenario does not separate the classes correctly. Points, which are most definitely no outliers (as they lie in groups) are classified wrong. The situation can be made better by raising $c$ to 100 and the degree $d$ to 3.
	
	We guess that almost all points are support vectors. Maybe some of the points inbetween other points of the same class are not.
	
\section*{Exercise 4}
Assume that $K1;K2 : \mathcal{X} \times \mathcal{X} \rightarrow \mathbb{R} $ are kernel functions. Which of the following functions are also a valid kernel? Prove or bring a counterexample.

A kernel function is defined as:\\
\[ \sum_{i,j=1}^n c_i c_j k(x_i,x_j) \geq 0  \] 
with $ x_i \in \mathcal{X}$, $c_1,...,c_n \in \mathbb{R}$, $n\geq 1$
\subsection*{a) $K = \alpha K_1 \ for \ \alpha \geq 0$ }
For the general kernel function we insert $\alpha K_1$ and move $\alpha$ in front of the sum.
\[\sum_{i,j=1}^n c_i c_j \alpha K_1(x_i,x_j) = \alpha \underbrace{\sum_{i,j=1}^n c_i c_j K_1(x_i,x_j)}_{\geq 0}   \]
As $\alpha \geq 0$ and $K_1 \geq 0$, is K a valid kernel.

\subsection*{b) $K = K_1 + K_2$}
As $K_1$ and $K_2$ are valid kernels, 
$ \sum_{i,j=1}^n c_i c_j K_1(x_i,x_j) \geq 0 $ and $ \sum_{i,j=1}^n c_i c_j K_2(x_i,x_j) \geq 0 $ holds.
With  
\begin{align*}
 \sum_{i,j=1}^n c_i c_j K_1(x_i,x_j) + \sum_{i,j=1}^n c_i c_j K_2(x_i,x_j) &= \underbrace{\sum_{i,j=1}^n c_i c_j (K_1(x_i,x_j) + K_2(x_i,x_j))}_{\geq 0} \\
 &= \underbrace{\sum_{i,j=1}^n c_i c_j K(x_i,x_j)}_{\geq 0}
\end{align*}



\section{Exercise 5}
 \end{document}