\documentclass[fontsize=12pt,a4paper]{scrartcl}
 
% Das Prozent\item Zeichen leitet einen Kommentar ein,
% es hilft ebenso, im Text Leerzeichen zu unterbinden.
 
% fontsize=12pt  Schriftgroesse in 10, 11 oder 12 Punkt
% a4paper        Papierformat ist hier A4
% landscape      Querformat wird natürlich unterstützt ;\item )
% parskip        Absatzabstand anstatt Einzüge
% draft          Der Entwurfsmodus deckt Schwächen auf
% {scrartcl}     Die Dokumentenklasse book, report, article
%                oder fürs deutsche scrbook, scrreprt, scrartcl
 
%\usepackage[ngerman]{babel} % Deutsche Sprachanpassungen
\usepackage[T1]{fontenc}    % Silbentrennung bei Sonderzeichen
\usepackage[latin1]{inputenc} % Direkte Angabe von Umlauten im Dokument.
                            % Wenn Sie an einem Mac sitzen,verwenden
                            % Sie ggf. „macce“ anstatt „utf8“.
 
\usepackage{textcomp}       % Zusätzliche Symbolzeichen
\usepackage{siunitx}
\usepackage{amsmath}
\usepackage{graphicx}
\usepackage{listings}
\usepackage{float}
\lstset{tabsize=4, showspaces=false}


\title{Machine Learning SS2013}
\subtitle{Norman Hendrich \\ Assignment 12}
\author{Arne Schr�der \and Falk Oswald \and Angel Bakardzhiev}
 
\date{\today}               % \today setzt das heutige Datum
 
\begin{document}
\maketitle                  % Titelei erzeugen
% \tableofcontents            % Inhaltsverzeichnis anlegen

\subsection*{Exercise 1: Sketch an improved algorithm that uses an incremental update of the state-value function.}

An idea: We can use a Bayes or Kalman filter to estimate the next state.

\subsection*{Exercise 2:  Why is $Q$-learning considered an \texttt{off - policy} control method?}

$Q$-learning is an \texttt{off-policy} method because it doesn't use a policy while learning. It uses only the current state $a_k$, in contrary the \texttt{SARSA} method, which is $on-policy$, uses also a feature state $a_{k+1}$ while learning and for the determination of this state $a_{k+1}$ the method needs a policy.

\end{document}