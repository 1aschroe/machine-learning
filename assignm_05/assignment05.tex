\documentclass[fontsize=12pt,a4paper]{scrartcl}
 
% Das Prozent-Zeichen leitet einen Kommentar ein,
% es hilft ebenso, im Text Leerzeichen zu unterbinden.
 
% fontsize=12pt  Schriftgroesse in 10, 11 oder 12 Punkt
% a4paper        Papierformat ist hier A4
% landscape      Querformat wird natürlich unterstützt ;-)
% parskip        Absatzabstand anstatt Einzüge
% draft          Der Entwurfsmodus deckt Schwächen auf
% {scrartcl}     Die Dokumentenklasse book, report, article
%                oder fürs deutsche scrbook, scrreprt, scrartcl
 
%\usepackage[ngerman]{babel} % Deutsche Sprachanpassungen
\usepackage[T1]{fontenc}    % Silbentrennung bei Sonderzeichen
\usepackage[latin1]{inputenc} % Direkte Angabe von Umlauten im Dokument.
                            % Wenn Sie an einem Mac sitzen,verwenden
                            % Sie ggf. „macce“ anstatt „utf8“.
 
\usepackage{textcomp}       % Zusätzliche Symbolzeichen
\usepackage{siunitx}
\usepackage{amsmath}
\usepackage{listings}
\lstset{tabsize=4, showspaces=false}


\title{Machine Learning SS2013}
\subtitle{Ulrike von Luxburg \\ Assignment 05}
\author{Arne Schr�der \and Falk Oswald \and Angel Bakardzhiev}
 
\date{\today}               % \today setzt das heutige Datum
 
\begin{document}
\maketitle                  % Titelei erzeugen
% \tableofcontents            % Inhaltsverzeichnis anlegen

 \section*{Exercise 1}
 
 \subsection*{Task}
Write the following linear program in the standard form by determining \textbf{A, b, c}.
\subsection*{Answer}

Substitute $x_3$ with $x_3' = -x_3$:

\begin{align*}
  \text{Minimize} \quad& x_1 -2x_2 -4x_3' \\
  \text{subject to}&\\
  -x_1+x_2 &\geq 1 \\
  3x1 - 2x_3' &\leq -1 \\
  -2x_1 + 5x_3' +4 &\leq 0\\
  x_1,x_2,x_3' &\leq 0
\end{align*}
\textbf{Standard form:}\\ \\
Minimize $c^Tx$ \\
subject to $Ax \leq b$ \\
and $x \leq 0$ \\
with 
\[
A=
\begin{bmatrix}
-1 & 1 & 0 \\
3 & 0 & -2 \\
-2 & 0 & 5 \\
\end{bmatrix},
b=
\begin{bmatrix}
1 \\
-1 \\
4 \\
\end{bmatrix},
c=
\begin{bmatrix}
1 \\
-2 \\
-4 \\
\end{bmatrix}
\]

\section*{Exercise 4}

\subsection*{Task}

For any $\gamma > 0$ the hyperplane defined by the solution $(w, b)$ to \textbf{(I)} $\min\limits_{w, b} \frac{1}{2} \|w\|^2$, subject to $y_i \cdot (\langle w, x_i \rangle -b) \geq 1$, is the same as the hyperplane defined by the solution $(w', b')$ defined by \textbf{(II)} $\min\limits_{w', b'} \frac{1}{2} \|w'\|^2$, subject to $y_i \cdot (\langle w', x_i \rangle - b') \geq \gamma$.

\subsection*{Answer}

For $\gamma' > 0$, minimizing $\frac{1}{2} \|w\|^2$ is equivalent to minimizing $\gamma' \cdot \frac{1}{2} \|w\|^2$. This especially applies for $\gamma' = \frac{1}{2} \gamma^{-2}$.

Given the problem \textbf{(II)} we know that
\begin{align*}
  & y_i \cdot (\langle w', x_i \rangle - b) \geq \gamma \\
  \Leftrightarrow & \gamma^{-1} \cdot y_i \cdot (\langle w', x_i \rangle - b') \geq 1 \\
  \Leftrightarrow & y_i \cdot (\langle \gamma^{-1} \cdot w', x_i \rangle - \gamma^{-1} \cdot b') \geq 1.
\end{align*}

We also can see that $\frac{1}{2} \|\gamma^{-1} \cdot w \|^2 = \frac{1}{2} \gamma^{-1} \cdot \frac{1}{2} \| w \|^2$. And using the assumption made at the beginning, we can conclude that \textbf{(II)} can be solved by the solution $(\gamma \cdot w, \gamma \cdot b)$.

Now we need to show that the two hyperplanes $H, H'$ defined by $(w, b)$ and $(\gamma \cdot w, \gamma \cdot b)$ coincide. Given a point $x \in H$ we know that 
\begin{align*}
\langle w, x \rangle + b &= 0 \\
\Leftrightarrow \langle \gamma \cdot w, x \rangle + \gamma \cdot b = \gamma \cdot (\langle w, x \rangle + b) &= \gamma \cdot 0 = 0.
\end{align*}

This means that also $x \in H'$ and because $x$ was arbitrary $H \subset H'$. And because the last implication is an equivalence, we know that $H = H'$.
\end{document}